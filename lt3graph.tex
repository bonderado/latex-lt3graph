%%%%%%%%%%%%%%%%%%%%%%%%%%%%%%%%%%%%%%%%%%%%%%%%%%%%%%%%%%%%%%%%%% \iffalse %%%%
%                                                                              %
%  Copyright (c) 2013 - Michiel Helvensteijn - www.mhelvens.net                %
%                                                                              %
%  https://github.com/mhelvens/latex-lt3graph                                  %
%                                                                              %
%  This work may be distributed and/or modified under the conditions           %
%  of the LaTeX Project Public License, either version 1.3 of this             %
%  license or (at your option) any later version. The latest version           %
%  of this license is in     http://www.latex-project.org/lppl.txt             %
%  and version 1.3 or later is part of all distributions of LaTeX              %
%  version 2005/12/01 or later.                                                %
%                                                                              %
%  This work has the LPPL maintenance status 'maintained'.                     %
%                                                                              %
%  The Current Maintainer of this work is Michiel Helvensteijn.                %
%                                                                              %
%  This work consists of the files lt3graph.tex and lt3graph.sty.              %
%                                                                              %
%%%%%%%%%%%%%%%%%%%%%%%%%%%%%%%%%%%%%%%%%%%%%%%%%%%%%%%%%%%%%%%%%%%%%%% \fi %%%%

\documentclass[a4paper]{lt3graph-packagedoc}

\usepackage       {amssymb}
\usepackage       {tikz}

\usetikzlibrary {matrix}

%%%%%%%%%%%%%%%%%%%%%%%%%%%%%%%%%%%%%%%%%%%%%%%%%%%%%%%%%%%%%%%%%%%%%%%%%%%%%%%%
% Setup                                                                        %
%%%%%%%%%%%%%%%%%%%%%%%%%%%%%%%%%%%%%%%%%%%%%%%%%%%%%%%%%%%%%%%%%%%%%%%%%%%%%%%%
 
\NoIndentAfterEnv{center} 
 
\moretexcs{
    \rightsquigarrow,
	\draw,
	\node,
	\cs_generate_variant:Nn,
	\int_eval:n,
	\cs_new:Nn,
	\clist_new:N,
	\clist_put_right:Nn,
	\graph_new:N,
	\graph_put_vertex:Nn,
	\graph_put_vertex:Nnn,
	\graph_put_edge:Nnn,
	\graph_put_edge:Nnnn,
	\graph_if_cyclic:NTF,
	\graph_map_edges_inline:Nn,
	\graph_map_topological_order_inline:Nn,
	\graph_map_vertices_inline:Nn,
	\graph_put_vertex:Nnf,
	\graph_get_degree:Nn,
	\graph_map_outgoing_edges_inline:Nnn,
	\graph_get_vertex:NnNT,
	\graph_put_edge:Nnnf,
	\graph_display_table:N,
	\graph_set_transitive_closure:NNNn,
	\graph_display_table:Nn,
	\graph_set_transitive_reduction:NN
}

%%%%%%%%%%%%%%%%%%%%%%%%%%%%%%%%%%%%%%%%%%%%%%%%%%%%%%%%%%%%%%%%%%%%%%%%%%%%%%%%
% Global Changes                                                               %
%%%%%%%%%%%%%%%%%%%%%%%%%%%%%%%%%%%%%%%%%%%%%%%%%%%%%%%%%%%%%%%%%%%%%%%%%%%%%%%%

\def\extratitlenote{
	The prefix \texttt{lt3} indicates that this package
	is a user-contributed \texttt{expl3} library, in contrast to
	packages prefixed with \texttt{l3}, which are officially supported
	by the \LaTeX3 team.
}

\changes{0.0.1}{2013/10/08}
  {initial version}
  
\changes{0.0.8}{2013/12/01}
  {a great many untracked changes}

\changes{0.0.9}{2013/12/07}
  {creation of the documentation}

\changes{0.1.0}{2014/04/13}
  {fixed a bug in\\
   \texttt{\textbackslash graph_(g)put_vertex} and\\
   \texttt{\textbackslash graph_(g)put_edge},
   which caused their global versions not to work properly}

%%%%%%%%%%%%%%%%%%%%%%%%%%%%%%%%%%%%%%%%%%%%%%%%%%%%%%%%%%%%%%%%%%%%%%%%%%%%%%%%
\begin{document}                                                               %
%%%%%%%%%%%%%%%%%%%%%%%%%%%%%%%%%%%%%%%%%%%%%%%%%%%%%%%%%%%%%%%%%%%%%%%%%%%%%%%%

\maketitle

\section {Introduction}  % % % % % % % % % % % % % % % % % % % % % % % % % % % %

This package provides a data-structure for use in the \LaTeX3 programming
environment. It allows you to represent a \emph{directed graph}, which
contains \emph{vertices} (nodes), and \emph{edges} (arrows) to
connect them.\footnote
	{ Mathematically speaking, a directed graph is a tuple
	  \( (V, E) \) with a set of vertices \( V \) and a set of edges
	  \( E \subseteq V \times V \) connecting those vertices. }
One such a graph is defined below:

\begin{latex-example}
\ExplSyntaxOn
    \graph_new:N         \l_my_graph
    \graph_put_vertex:Nn \l_my_graph {v}
    \graph_put_vertex:Nn \l_my_graph {w}
    \graph_put_vertex:Nn \l_my_graph {x}
    \graph_put_vertex:Nn \l_my_graph {y}
    \graph_put_vertex:Nn \l_my_graph {z}
    \graph_put_edge:Nnn  \l_my_graph {v} {w}
    \graph_put_edge:Nnn  \l_my_graph {w} {x}
    \graph_put_edge:Nnn  \l_my_graph {w} {y}
    \graph_put_edge:Nnn  \l_my_graph {w} {z}
    \graph_put_edge:Nnn  \l_my_graph {y} {z}
    \graph_put_edge:Nnn  \l_my_graph {z} {x}
\ExplSyntaxOff
\end{latex-example}

Each vertex is identified by a \emph{key}, which, to this library, is a string:
a list of characters with category code 12 and spaces with category code 10.
An edge is then declared between two vertices by referring to their keys.

We could then, for example, use TikZ to draw this graph:

\begin{latex-example-show}
\centering
\begin{tikzpicture}[every path/.style={line width=1pt,->}]
    \newcommand{\vrt}[1]{ \node(#1){\ttfamily\vphantom{Iy}#1}; }
    \matrix [nodes={circle,draw},
             row sep=1cm, column sep=1cm,
             execute at begin cell=\vrt] {
        v  &  w  &  x  \\
           &  y  &  z  \\ };
    \ExplSyntaxOn
        \graph_map_edges_inline:Nn \l_my_graph
            { \draw (#1) to (#2); }
    \ExplSyntaxOff
\end{tikzpicture}
\end{latex-example-show}

Just to be clear, this library is \emph{not about drawing} graphs.
It does not, inherently, understand any TikZ.
It is about \emph{representing} graphs. This allows us do perform analysis
on their structure. We could, for example, determine if there is
a cycle in the graph:

\begin{latex-example-show}
\ExplSyntaxOn
    \graph_if_cyclic:NTF \l_my_graph {Yep} {Nope}
\ExplSyntaxOff
\end{latex-example-show}

Indeed, there are no cycles in this graph. We can also list
its vertices in topological order:

\begin{latex-example-show}
\ExplSyntaxOn
    \clist_new:N \LinearClist
    \graph_map_topological_order_inline:Nn \l_my_graph
        { \clist_put_right:Nn \LinearClist {\texttt{#1}} }
\ExplSyntaxOff
Visiting dependencies first: \( \LinearClist \)
\end{latex-example-show}

There is a great deal more that can be done with graphs
(some of which is even implemented in this library). A
common use-case will be to attach data to vertices and/or
edges. You could accomplish this with a property map from
|l3prop|, but this library has already done that for you!
Every vertex and every edge can store arbitrary token lists.\footnote
    {This makes the mathematical representation of our graphs
     actually a 4-tuple \( (V, E, v, e) \), where \( v: V \to T\!L \)
     is a function that maps every vertex to a token list and
     \( e: E \to T\!L \) is a function that maps every edge (i.e., pair of vertices)
     to a token list.}

In the next example we store the \emph{degree} (the number
of edges, both incoming and outgoing) of each vertex
inside that vertex as data. We then
query all vertices directly reachable from \texttt{w}
and print their information in the output stream:

\begin{latex-example}
\ExplSyntaxOn
    \cs_generate_variant:Nn \graph_put_vertex:Nnn {Nnf}
    \graph_map_vertices_inline:Nn \l_my_graph {
        \graph_put_vertex:Nnf \l_my_graph {#1}
        	{ \graph_get_degree:Nn \l_my_graph {#1} }
    }
\ExplSyntaxOff
\end{latex-example}

It's just an additional parameter on the |\graph_put_vertex| function.
Edges can store data in the same way:

\begin{latex-example}
\ExplSyntaxOn
    \graph_map_edges_inline:Nn \l_my_graph {
        \graph_put_edge:Nnnn \l_my_graph {#1} {#2}
            { \int_eval:n{##1 * ##2} }
    }
\ExplSyntaxOff
\end{latex-example}

The values |##1| and |##2| represent the data stored in,
respectively, vertices |#1| and |#2|. This is a feature
of |\graph_put_edge:Nnnn| added for your convenience.

\needspace{2\baselineskip}
We can show the resulting graph in a table, which is handy for debugging:

\begin{latex-example-show}
\ExplSyntaxOn \centering
    \graph_display_table:N \l_my_graph
\ExplSyntaxOff
\end{latex-example-show}

\fboxsep=1pt
The \fcolorbox{black}{green}{\vphantom{Ig}green} cells represent edges
directly connecting two vertices. The \fcolorbox{black}{green!40!yellow}{\vphantom{Ig}\tiny (tr)}
cells don't have edges, but indicate that there is a sequence of edges connecting
two vertices transitively.

Two vertices can have at most two arrows connecting them:
one for each direction. If you want to represent a \emph{multidigraph}
(or \emph{quiver}; I'm not making this up),
you could consider storing a (pointer to a) list at each edge.

Finally, we demonstrate some transformation functions.
The first generates the transitive closure of a graph:

\begin{latex-example}
\ExplSyntaxOn
    \graph_new:N \l_closed_graph
    \cs_new:Nn \__closure_combiner:nnn { #1,~#2,~(#3) }
    \graph_set_transitive_closure:NNNn
        \l_closed_graph \l_my_graph
        \__closure_combiner:nnn {--}
\ExplSyntaxOff
\end{latex-example}
\begin{latex-example-show}
\ExplSyntaxOn \centering
    \graph_display_table:N \l_my_graph
    \(\ \ \rightsquigarrow\ \ \)
    \graph_display_table:Nn \l_closed_graph
        { row_keys = false, vertex_vals = false }
\ExplSyntaxOff
\end{latex-example-show}

There is a simpler version (|\graph_set_transitive_closure:NN|)
that sets the values of the new edges to the empty token-list.
The demonstrated version takes an expandable function to determine the new value,
which has access to the values of the two edges being combined (as |#1| and |#2|),
as well as the value of the possibly already existing transitive edge (as |#3|).
If there was no transitive edge there already, the value passed as |#3| is
the fourth argument of the transformation function; in this case |--|.

The second transformation function generates the transitive reduction:

\begin{latex-example}
\ExplSyntaxOn
    \graph_new:N \l_reduced_graph
    \graph_set_transitive_reduction:NN
        \l_reduced_graph \l_my_graph
\ExplSyntaxOff
\end{latex-example}
\begin{latex-example-show}
\ExplSyntaxOn \centering
    \graph_display_table:N \l_my_graph
    \(\ \ \rightsquigarrow\ \ \)
    \graph_display_table:Nn \l_reduced_graph
        { row_keys = false, vertex_vals = false }
\ExplSyntaxOff
\end{latex-example-show}



\section {API Documentation}  % % % % % % % % % % % % % % % % % % % % % % % % %

Sorry! There is no full API documentation yet. But in the meantime, much of the API
is integrated in the examples of the previous section, and everything is
documented (however sparsely) in the implementation below.



%%%%%%%%%%%%%%%%%%%%%%%%%%%%%%%%%%%%%%%%%%%%%%%%%%%%%%%%%%%%%%%%%%%%%%%%%%%%%%%%
\end{document}                                                                 %
%%%%%%%%%%%%%%%%%%%%%%%%%%%%%%%%%%%%%%%%%%%%%%%%%%%%%%%%%%%%%%%%%%%%%%%%%%%%%%%%
